%\RequirePackage{luatex85}
\documentclass[10pt]{article}
\usepackage{geometry}
\usepackage{cancel}
\usepackage{graphicx}
\usepackage{sectsty}
\pagenumbering{gobble}
\geometry{papersize={10in,2.3in}, left= 0.1in,
	right=0.1in,
	top=0.1in,
	bottom=0.1in}
%sectionfont{\centering}
%\geometry{
%	letterpaper,
%	left= 1in,
%	right=1in,
%	top=20mm
%}
\usepackage{amssymb,amsmath}
\begin{document}
	\nopagebreak
\subsection*{\textbf{\(R^2 \)}}
The model has \(R^2\approx 0.5\).  This means that our model explains about 50\% of the variation by using {\it sqft\_living} as independent variable.
\subsection*{ANOVA}
Is our model with many explanatory variable better than the model with zero explanatory variables?\\
Our model has \(F-statistic = 1.737\times 10^4\)  and \(Prob > F\) is \(0.000\).\\
\textbf{The Null Hypothesis:   The slope\(=0\)\\
The Alternative Hypothesis: The slope\(\ne 0\)\\
}
Our p-value for this model is \(p=0.000 < 0.05 = \alpha\). Thus, we have enough evidence to reject the Null Hypothesis at $5\%$ level of significance and we conclude that the Test  tells us, that at least one of the coefficients is not $0$. Since our p-value is $0$, there is a $0\%$ probability that the improvements that we are seeing with our independent variables model are due to random chance alone.
\end{document}
