%\RequirePackage{luatex85}
\documentclass[10pt]{article}
\usepackage{geometry}
\usepackage{cancel}
\usepackage{graphicx}
\usepackage{sectsty}
\pagenumbering{gobble}
\geometry{papersize={10in,12.5in}, left= 0.1in,
	right=0.1in,
	top=0.1in,
	bottom=0.1in}
%sectionfont{\centering}
%\geometry{
%	letterpaper,
%	left= 1in,
%	right=1in,
%	top=20mm
%}
\usepackage{amssymb,amsmath}
\begin{document}
	\nopagebreak
\section*{Model}
\begin{align*}
	\ln(price) =& 10.2082 + 0.3618\cdot \text{waterfront} - 0.0160\cdot\text{bedrooms} -0.0153 \cdot\text{bathrooms} +0.1400\cdot\text{sqft\_living}^{0.3} \\ +& 0.0088 \cdot \text{floors}
	+ 0.1494\cdot\text{view}^{0.5}+ 0.0105\cdot\text{grade}^2+0.1187\cdot\ln\left(\text{sqft\_living15}\right)
\end{align*}
\section*{Explanation of the Model}

 Before I begin explain the coefficients, I notice that $P>|t|$ for the {\it floors} variable is $0.155$, which makes  {\it floors} insignificant for the analysis. 
 \begin{enumerate}
 	\item  The model has the sample intercept of $10.2082$.
If we assume that all explanatory variables are zeros, this would mean that the price would be $e^{10.2082}\approx 27,124$
 \item  $0.3618$ is the coefficient for *waterfront*.
 {\it Waterfront} is a categorical variable coded as $0$ or $1$, a one unit difference represents switching from one category to the other. $10.2082$ is then the average difference in  {\it price} between the category for which  {\it waterfront} = \(0\) (no waterfront) and the category for which  {\it waterfront} = \(1\) (the house has a waterfront). So compared to $\ln(\text{\it price})$ of the house with no waterfront, we would expect the $\ln(\text{\it price})$ for the house with waterfront to be $0.3618$ higher, on average, if we fix all other explanatory variables.\\
Let $y_2 = \text{price of the house with waterfront}$   and $y_1 = \text{price of the house with no waterfront}$, then 
$$
\large
\ln y_2 - \ln y_1 = 0.3618 \implies \ln \frac{y_2}{y_1} = 0.3618 \implies \frac{y_2}{y_1} = e^{0.3618}$$
To understand the change in *price* in percents, if we switch from the house with no waterfront to the house with waterfront while keeping all other variables the same, will use the following formula:
$$
\large
100\times \left[\frac{y_2-y_1}{y_1}\right]= 100\times \left[\frac{y_2}{y_1} - 1\right] = 100\times \left[ e^{0.3618} - 1\right] \approx 43.59\%$$

Thus, switching from the house with no waterfront to the house with waterfront while keeping all other explanatory variables fixed, will increase the price by 43.59\%.

 \item  $-0.0160$ is the coefficient for number of {\it bedrooms}. Let $y_2 = \text{price}$ for the house with $x_2$ number of bedrooms and $y_1 = \text{price}$ for the house with $x_1$ number of bedrooms, then 
$$
\large
\ln y_2 - \ln y_1 = -0.016 x_2 - (-0.016 x_1) = -0.016 (x_2 - x_1) \implies \ln\frac{y_2}{y_1} = -0.016 (x_2-x_1)\implies \frac{y_2}{y_1} = e^{-0.016(x_2-x_1)}$$

If we increase the number of bedrooms by 1 while keeping the other variables fixed and use percents, we will get the following
$$
\large
100 \times \left[\frac{y_2}{y_1} - 1 \right] = 100 \times \left[e^{-0.016} -1 \right] \approx -1.58\%
$$
Thus, increasing the number of bedrooms by 1 while keeping the other variables fixed, will decrease the price of the house by 1.58\%. Which is a strange result.
\item The same explanation we have for $-0.0153$ which a coefficient for number of bathrooms. If we increase the number of bathrooms by 1 while keeping the other variables fixed, will decrease the price of the house by 1.51\%. Which is a strange result.
 \item $0.1400$ is the coefficient for {\it sqft\_living}.
Let $y_2 = \text{price}$ for the house with $x_2$ {\it sqft\_living} and $y_1 = \text{price}$ for the house with $x_1$ {\it sqft\_living}, then 
$$
\large
\ln y_2 - \ln y_1 = 0.1400 x_2^{0.3} - 0.1400 x_1^{0.3} \implies \ln \frac{y_2}{y_1} = 0.1400 \left( x_2^{0.3} - x_1^{0.3}\right)\implies \frac{y_2}{y_1}=e^{0.1400 \left( x_2^{0.3} - x_1^{0.3}\right)}
$$
If we increase the {\it sqft\_living} from $1000ft$ to $1100ft$ while keeping the other variables fixed, we will get the following change in price in percents:
$$
\large
100 \times \left[\frac{y_2}{y_1} -1  \right] = 100 \times \left[ e^{0.1400 \left( 1100^{0.3} - 1000^{0.3}\right)} - 1 \right]\approx 3.28\%
$$
If the {\it sqft\_living} is $1000ft$ and we increase it to $1100ft$ while keeping the other variables fixed, we will get the change in price of $3.28\%$. In this particular example $10\%$ change in {\it sqft\_living} starting from $x_1 = 1000ft$ forces $3.28\%$ change in price.
\item  $0.1494$ is the coefficient for the *view*. The coefficient for the {\it view} has the same explanation as the {\it sqft\_living}. If we increase the {\it view} by $1$ unit from $2$ to $3$ while keeping the other variables fixed, we will get the following:
$$
\large 
100 \times \left[\frac{y_2}{y_1} -1  \right] = 100 \times \left[ e^{0.1494 \left( 3^{0.5} - 2^{0.5}\right)} - 1 \right]\approx 4.86\%
$$
In this particular example if the {\it view} will increase from $2$ to $3$, the price will increase by 4.86\%.

\item $0.0105$ is the coefficient for the {\it grade}.  The coefficient for the *grade* has the same explanation as the {\it sqft\_living}. If we increase the {\it grade} by $1$ unit from $2$ to $3$ while keeping the other variables fixed, we will get the following:
$$
\large 
100 \times \left[\frac{y_2}{y_1} -1  \right] = 100 \times \left[ e^{0.0105\left( 3^{2} - 2^{2}\right)} - 1 \right]\approx 5.39\%
$$
In this particular example if the {\it grade} will increase from $2$ to $3$ while other variables stay the same, the price will increase by 5.39\%.
 \item $0.1187$ is the coefficient for the {\it sqft\_living15}. Let $y_2 = \text{price}$ for the house with $x_2$ {\it sqft\_living15} and $y_1 = \text{price}$ for the house with $x_1$ {\it sqft\_living15}, then 
$$
\large
\ln y_2 - \ln y_1 = 0.1187 \ln x_2 - 0.1187 \ln x_1 \implies \ln \frac{y_2}{y_1} = 0.1187 \ln \frac{x_2}{x_1} \implies \frac{y_2}{y_1} = e^{0.1187}\frac{x_2}{x_1}
$$
The change in price in precent will be:
$$
\large
100 \times \left[\frac{y_2}{y_1} - 1 \right]= 100 \times \left[ e^{0.1187}\frac{x_2}{x_1} - 1 \right]
$$
If we increase the {\it sqft\_living15} by $100$ units from $1000$ to $1100$ while keeping the other variables fixed, we will get the following:
$$
\large
100 \times \left[ e^{0.1187}\times \frac{1100}{1000} - 1 \right] \approx 23.87\%
$$
Thus, 10\% increase in {\it sqft\_living15} will lead to 23.87\% increase in {\it price}.
 \end{enumerate}
\end{document}
