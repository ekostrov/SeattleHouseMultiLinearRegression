%\RequirePackage{luatex85}
\documentclass[10pt]{article}
\usepackage{geometry}
\usepackage{cancel}
\usepackage{graphicx}
\usepackage{sectsty}
\pagenumbering{gobble}
\geometry{papersize={8in,7in}, left= 0.1in,
	right=0.1in,
	top=0.1in,
	bottom=0.1in}
%sectionfont{\centering}
%\geometry{
%	letterpaper,
%	left= 1in,
%	right=1in,
%	top=20mm
%}
\usepackage{amssymb,amsmath}
\begin{document}
	\nopagebreak
\section*{Model}
	\[\hat{y}=11.9524+0.0029\cdot \hat{x},\quad \text{where}\quad  \hat{y}=\ln (\text{price})\text{ and } \hat{x} = \text{sqft\_living}^{0.78}  \]
\section*{Explanation of the Model}
\subsection*{Intercept and slope}
The model has the sample intercept of \(11.9524\) and the slope of \(0.0029\).  To interpret the slope, we have to transform \(\hat{x}\) and \(\hat{y}\) towards original {\it sqft\_living} and {\it price}. Let \(x\) = {\it sqft\_living}  and \(y\) = {\it price} for the derivation. We have  \(\hat{x} = x^{0.78}\) and \(\hat{y} = \ln (y)\implies y  = e^{\hat{y}}\). Suppose the original sqft\_living is \(x_1\) and it moved up to \(x_2\), then we have the following :
\begin{align*}
	y_1 =& e^{\hat{y}_1 }= e^{11.9524+0.0029\cdot x_1^{0.78}}  = e^{11.9524}\cdot e^{0.0029\cdot x_1^{0.78}}\\
	y_2 =& e^{\hat{y}_2 }= e^{11.9524+0.0029\cdot x_2^{0.78}}  = e^{11.9524}\cdot e^{0.0029\cdot x_2^{0.78}}
\end{align*}
To understand the change in {\it price} in percents we will use the following formula:

\begin{align*}
100 \times \left[\frac{y_2-y_1}{y_1} \right] =& 100 \times \left[\frac{y_2}{y_1} - 1\right] = 100\times \left[\frac{ \cancel{e^{11.9524}}\cdot e^{0.0029\cdot x_2^{0.78}}}{\cancel{e^{11.9524}}\cdot e^{0.0029\cdot x_1^{0.78}}} -1 \right] \\
=& 100 \times \left[ e^{0.0029\left( x_2^{0.78} - x_1^{0.78} \right)} -1 \right]
\end{align*}
For example, if the {\it sqft\_living} is \(1000ft\) and we increase it to \(1100\), we will get the change in price of \(100\times \left[e^{0.0029(1100^{0.78}-1000^{0.78})}-1\right] \approx 5.02\% \). In this particular example \(10\%\) change in sqft\_living starting from \(x_1 = 1000ft\) forces \(5.02\%\) change in price.
\subsection*{\textbf{\(R^2 \)}}
The model has \(R^2\approx 0.45\).  This means that our model explains about 45\% of the variation by using {\it sqft\_living} as independent variable.
\subsection*{ANOVA}
Is our model with one explanatory variable better than the model with zero explanatory variables?\\
Our model has \(F-statistic = 1.737\times 10^4\)  and \(Prob > F\) is \(0.000\).\\
\textbf{The Null Hypothesis:   The slope\(=0\)\\
The Alternative Hypothesis: The slope\(\ne 0\)\\
}
Our p-value for this model is \(p=0.000 < 0.05 = \alpha\). Thus, we have enough evidence to reject the Null Hypothesis at \(5\%\) level of significance and we conclude that the Test  tells us, that our slope is not \(0\). Since our p-value is \(0\), there is a \(0\%\) probability that the improvements that we are seeing with our one independent variable model are due to random chance alone.
\end{document}
